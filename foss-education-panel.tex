\documentclass{sig-alt-release2}
\usepackage{comment}
\usepackage{url}
\begin{document}

% --- Author Metadata here ---
\conferenceinfo{SIGCSE'11,} {March 9--12, 2011, Dallas, Texas, USA.}
\CopyrightYear{2011}
\crdata{978-1-4503-0500-6/11/03}
\clubpenalty=10000
\widowpenalty = 10000% --- End of Author Metadata ---

%\title{Open Source Software Education}
\title{Learning Through Open Source Participation}
%\subtitle{Opportunities and Challenges}

\numberofauthors{4}  
        
\author{     
% 1st. author
\alignauthor                   
Heidi J. C. Ellis\\
       \affaddr{Western New England College}\\
       \affaddr{1215 Wilbraham Road}\\
       \affaddr{Springfield, MA 01119}\\
       \email{hellis@wnec.edu}  \\
% 2nd. author
\alignauthor Mel Chua\\
       \affaddr{Red Hat}\\
       \affaddr{1801 Varsity Drive}\\
       \affaddr{Raleigh, North Carolina 27606}\\
       \email{mel@redhat.com}  
			\and
			% 3rd. author
			\alignauthor  Matthew C. Jadud\\  
			       \affaddr{Allegheny College}\\
			       \affaddr{520 N. Main St.}\\
			       \affaddr{Meadville, PA 16335}\\
			       \email{matthew.c@jadud.com}    
			\alignauthor  Gregory W. Hislop (Moderator)\\
			       \affaddr{Drexel University}\\
						 \affaddr{3141 Chestnut St.}
			       \affaddr{Philadelphia, PA 19104}\\
			       \email{hislop@drexel.edu}    
} 

\begin{comment} 

	\end{comment}
	
\maketitle

\category{K.3.2}{Computers and Education}{Computer and Information Science Education -- {\em Computer Science Education}}
\terms{Human Factors}

\keywords{Open source software, faculty development, education}

\section{Summary}
There are many mature free open source software (FOSS) and documentation projects that provide excellent learning opportunities for students.  In the context of active learning, communities that surround open source projects are particularly interesting as they are transparent meritocracies that allow students to observe, study, and, best of all, contribute as part of their learning.  In exploring the learning opportunities that come from students engaging in open source projects, educators also believe that these projects are intrinsically motivating. They provide an opportunity for ongoing self-regulated learning in the context of working on large-scale projects, to interact with experienced developers and content creators, and to see meaningful, real-world outcomes as a result of their efforts.

This panel will present four different perspectives on student involvement in FOSS projects. Matt Jadud will present a variety of different ways that students can contribute to FOSS projects beyond code including writing, communicating, bug fixing, and more. Mel Chua will present the industry perspective on student involvement in FOSS projects including motivation for the FOSS community to be open to students as well as the challenges presented by student involvement. Heidi Ellis will discuss how Humanitarian FOSS can provide a welcoming environment for student learning including identifying some of the characteristics of more approachable FOSS projects. Greg Hislop will present the viewpoint of faculty involving the students in FOSS including a discussion of barriers to faculty involvement and how such barriers can be overcome. 

\section{Heidi Ellis -- Humanitarian FOSS}
FOSS projects with a humanitarian nature, Humanitarian FOSS or HFOSS, have been shown to motivate students in pursuing computing degrees while providing them with the opportunity to better the human condition in some fashion. Students across all levels of computing experience find participating in an HFOSS project motivating and students with less computing background indicate that participating in an HFOSS project has caused them to consider computing as a major or at least taking further computing courses. However, the ease of involving students in HFOSS projects varies across communities. 

Heidi will discuss the factors that impact the ease or difficulty with which students become involved in an HFOSS project. She will report on experiences of students in a senior-level Software Engineering course at Western New England College. Students are working on Caribou which is a GNOME Accessibility project. Caribou is an onscreen keyboard that is accessible to users needing to interact with the computer using a pointing device. Students are involved in the GNOME Accessibility community and are investigating ways to make the application more robust. As part of the course, students in the Software Engineering course will introduce freshman Computer Science and Information Technology majors to HFOSS and hopefully generate excitement in computing in our new majors.

In addition to the Software Engineering class working on the Caribou project, there are several Western New England College students who are participating in other HFOSS projects as part of independent study courses. These students are both CS and IT majors and are exploring ways that both groups can contribute to HFOSS projects. Heidi will report on the experiences of these students with joining and interacting with their respective HFOSS communities.  

\newpage

\section{Matt Jadud -- Beyond Computing}

Large, open projects provide opportunities for what Jean Lave and Etienne Wenger describe as {\em legitimate peripheral participation}. This means that it is possible for a classroom full of students studying computing (or literature or psychology) to contribute meaningfully to a project with hundreds or even thousands of contributors by beginning their work on the edge of a community (perhaps by fixing typos in a wiki page or reporting bugs), and if they continue contributing, they can grow naturally into increasingly central roles within the project. There is great power in immersing a class in a decentralized community of contributors, and it is our responsibility as computing educators to introduce students to the world of open collaboration.

In computing, open projects force students to communicate and collaborate in ways that traditional pedagogic approaches eschew: we get teamwork for free, and that matters. In the context of robotics and embedded systems work at Allegheny College, students have contributed to open software and documentation projects (like the work done on the Flying Gator UAV and a Creative Commons licensed text regarding parallel programming for for embedded systems). For example, in our course regarding human-centered design, students work directly with open communities to improve the end-user's experience of living, shipping software\footnote{\url{http://rockalypse.org/courses/cmpsc303f10/}}. Students can write documentation, conduct interviews, design logos, all while using mailing lists, wikis, and IRC to collaborate with open source communities. 

\section{Mel Chua -- FOSS Community}

Red Hat, Fedora, and other open communities have worked on ``on-ramping'' student into FOSS participation using projects like the POSSE workshop for professors\footnote{\url{http://teachingopensource.org/index.php/POSSE}}, summer camps for high school students, undergraduate co-op mentorships, and more. Through working with the professors in the Teaching Open Source community, we've come to recognize some commmon elements of successful engagements for bridging the FOSS and academic worlds.

First, a successful engagement must focus on helping students become independent do-ers. Students need to learn not just how to use tools, but how to join a community of toolmakers. FOSS communities are full of self-sufficient and intrinsically motivated makers, and the artifacts they create as they learn (chat logs, blog posts, code and code reviews) are openly licensed and freely available so students can ``learn to do'' from (and alongside) role models beyond their campus context, with professors alongside to help them analyze and reflect on the experiences they're encountering. In the FOSS world, we call this being {\em productively lost}.

Second, the engagement must immerse students in whole-project thinking, seeing where their individual contributions fit into a large and complex system. Open source projects operate in the real world, where large projects with distributed and diverse teams rely on many types of people---project management, design, marketing, testing, support, translation, documentation,  and so forth---in order to get a software product to make a difference in the lives of its users. Each of these roles is one that students can step into from day one. As they do so, the community helps guide them through the process of negotiating technical and social interfaces with other groups on the project that fill diverse roles and may work in dramatically different contexts.

Third, the people involved in the engagement need to make sure that learning and insight flows in both directions. The academic world can benefit from the authentic learning experiences and transparency of open source communities, but open source contributors also have a tremendous amount to learn from students and professors, such as how to reflect on their own growth as learners. FOSS projects, with their ``sink or swim'' mentality, are notoriously bad at helping newcomers participate. The diversity and retention rates of our communities could be much improved by thinking critically about the participatory scaffolding we build.  This is exactly what educators do year after year, getting students established as self-sufficient new contributors to a discipline that may initially be unfamiliar to them.

\section{Greg Hislop -- Faculty Adoption}        
It is generally easy to convince faculty that student participation in FOSS offers excellent learning opportunities.  However, faculty belief in the potential value of FOSS participation will not necessarily cause them to incorporate FOSS activities into their courses.  There are a variety of issues and obstacles that may hinder adoption.

To begin, FOSS participation may require that faculty master new material.  Most likely, the faculty member will need to learn about the FOSS project.  They may also need to learn about development processes and tools used by the FOSS project.  Even if faculty are comfortable with commercial software development, they may not be familiar with the particular approaches used in FOSS projects.  The total amount to learn can be considerable, and faculty time for professional development is scarce.

Second, FOSS participation requires interacting with a FOSS project community.  These communities are often globally distributed, and consist of a mix of full-time professionals and part-time, volunteer participants.  In addition to taking time, community dependence and interaction introduces uncertainty into the course that may make faculty uncomfortable.  It also requires mutual adjustment in schedule and expectations as participants figure out how to map a fixed length term to a FOSS project schedule.

Third, the students in a particular class section represent a varied workforce.  Knowledge, skill, and ability to learn and self-direct typically vary greatly across a set of students. In addition, student interest and motivation for a particular course can also impact results.  Faculty members need to not only grade student results, but most likely will need to manage student contribution to the FOSS project to ensure that contribution quality is sufficient.

\section{Acknowledgments}
This material is based on work supported by the National Science Foundation under Grant Nos. DUE-0736874, DUE-0722137, DUE-0940925, and DUE-0940893 Any opinions, findings and conclusions or recommendations expressed in this material are those of the author(s) and do not necessarily reflect the views of the National Science Foundation (NSF).

\end{document}
